\input{../../permve-ntnu-latex/assignment.tex}

\title{	
\normalfont \normalsize 
\textsc{Norwegian University of Science and Technology\\TDT4171 -- Artificial Intelligence Methods} \\ [25pt]
\horrule{0.5pt} \\[0.4cm]
\huge Assignment 1 \\
\horrule{2pt} \\[0.5cm]
}

\renewcommand\thesection{\Roman{section}}
\renewcommand{\thesubsection}{\thesection.\arabic{section}}

\makeatletter
\renewcommand{\@seccntformat}[1]{\csname the#1\endcsname\quad}
\makeatother

\author{Per Magnus Veierland\\permve@stud.ntnu.no}

\date{\normalsize\today}

\begin{document}
\maketitle

\section{Counting and basic laws of probability}

\subsection{5-card Poker Hands}

\textbf{Consider the domain of dealing 5-card poker hands from a standard deck of 52 cards, under the assumption that the dealer is fair.}

\begin{enumerate}[label=\alph*)]
\item \textbf{How many atomic events are there in the joint probability distribution\\(i.e., how many 5-card hands are there)?}\\
The number of 5-card hands in poker can be found with combinatorics by noting that the ordering of the cards do not matter:
\begin{displaymath}
\binom{52}{5} =
\frac{52!}{5!(52 - 5)!} =
\frac{52 \cdot 51 \cdot 50 \cdot 49 \cdot 48}{5 \cdot 4 \cdot 3 \cdot 2 \cdot 1} =
\frac{311875200}{120} =
2598960
\end{displaymath}
\item \textbf{What is the probability of each atomic event?}\\
The probability of each atomic event (being dealt a distinct poker hand) is:
\begin{displaymath}
\frac{1}{2598960} \approx 3.84769 \cdot 10^{-7}
\end{displaymath}
\item
\begin{itemize}
\item \textbf{What is the probability of being dealt a royal straight flush?}\\
A royal straight flush consists of five cards of the same suit in a sequence, where the top card is an ace. There exists four possible royal straight flush hands; one for each suit. The probability of being dealt a royal straight flush is:
\begin{displaymath}
\frac{4}{2598960} \approx 1.53908 \cdot 10^{-6}
\end{displaymath}
\item \textbf{What is the probability of being dealt four of a kind?}\\
Four of a kind is a poker hand where four of the cards have the same rank, and the fifth card can be any other rank and suit. The number of possible four of a kind hands is the number of ranks; 13, multiplied by the number of possible fifth cards; which is any of the twelve remaining ranks from one of four suits:
\begin{displaymath}
\frac{13 \cdot 12 \cdot 4}{2598960} \approx 0.00024
\end{displaymath}
\end{itemize}
\end{enumerate}

\end{document}

