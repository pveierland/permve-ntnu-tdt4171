\input{../../permve-ntnu-latex/assignment.tex}

\title{	
\normalfont \normalsize 
\textsc{Norwegian University of Science and Technology\\TDT4171 -- Artificial Intelligence Methods} \\ [25pt]
\horrule{0.5pt} \\[0.4cm]
\huge Assignment 1 \\
\horrule{2pt} \\[0.5cm]
}

\author{Per Magnus Veierland\\permve@stud.ntnu.no}

\date{\normalsize\today}

\begin{document}
\maketitle

\section{Counting and basic laws of probability}

\subsection{5-card Poker Hands}

\textit{Consider the domain of dealing 5-card poker hands from a standard deck of 52 cards, under the assumption that the dealer is fair.}

\begin{enumerate}[label=\alph*)]
\item \textit{How many atomic events are there in the joint probability distribution\\(i.e., how many 5-card hands are there)?}\\
The number of 5-card hands in poker can be found with combinatorics by noting that the ordering of the cards do not matter:
\begin{equation}
\binom{52}{5} =
\frac{52!}{5!(52 - 5)!} =
\frac{52 \cdot 51 \cdot 50 \cdot 49 \cdot 48}{5 \cdot 4 \cdot 3 \cdot 2 \cdot 1} =
\frac{311875200}{120} =
2598960
\end{equation}
\item \textit{What is the probability of each atomic event?}\\
The probability of each atomic event, i.e. the probability of any distinct poker hand, is:
\begin{equation}
\frac{1}{2598960} \approx 3.84769 \cdot 10^{-7}
\end{equation}
\item
\begin{itemize}
\item \textit{What is the probability of being dealt a royal straight flush?}\\
A royal straight flush consists of five cards of the same suit in a sequence, where the top card is an ace. There exists four possible royal straight flush hands; one for each suit. The probability of being dealt a royal straight flush is:
\begin{equation}
\frac{4}{2598960} \approx 1.53908 \cdot 10^{-6}
\end{equation}
\item \textit{What is the probability of being dealt four of a kind?}\\
Four of a kind is a poker hand where four of the cards have the same rank, and the fifth card can be any other rank and suit. The number of possible four of a kind hands is the number of ranks; 13, multiplied by the number of possible fifth cards; which is any of the twelve remaining ranks from one of four suits:
\begin{equation}
\frac{13 \cdot 12 \cdot 4}{2598960} \approx 0.00024
\end{equation}
\end{itemize}
\end{enumerate}

\subsection{Two cards in a deck}

\textit{Two cards are randomly selected from a deck of 52 playing cards.}

\begin{enumerate}[label=\alph*)]
\item \textit{What is the probability they constitute a pair (that is, that they are the same denomination)?}\\
The number of pairs in a deck of 52 playing cards is given by the number of ranks times the number of ways to form a pair from four suits within a rank:
\begin{equation}
\label{eq:number_of_pairs}
13 \cdot \binom{4}{2} =
13 \cdot \frac{4!}{2!(4 - 2)!} =
13 \cdot \frac{4 \cdot 3}{2} =
13 \cdot \frac{12}{2} =
78
\end{equation}
The number of ways to draw two cards from a deck of 52 playing cards is given by:
\begin{equation}
\binom{52}{2} =
\frac{52!}{2!(52 - 2)!} =
\frac{52 \cdot 51}{2} =
1326
\end{equation}
The probability that drawing two cards from a deck of 52 playing cards yields a pair is given by the number of pairs in a deck, divided by the number of ways to draw two cards from a deck:
\begin{equation}
\frac{78}{1326} \approx 0.05882
\end{equation}
The solution can also be found by reasoning that the probability of the second card drawn will be of the same rank as the first card drawn is given by:
\begin{equation}
\frac{4 - 1}{52 - 1} = \frac{3}{51} \approx 0.05882
\end{equation}
\item \textit{What is the conditional probability they constitute a pair given that they are of different suits?}\\
The number of ways to draw two cards of different suits from a deck is given by the number of ways to draw two cards from a deck, minus the number of suits times the number of ways to draw two cards from the same suit:
\begin{equation}
\binom{52}{2} - 4 \cdot \binom{13}{2} =
\frac{52!}{2!(52 - 2)!} - 4 \cdot \frac{13!}{2!(13 - 2)!} =
\frac{52 \cdot 51}{2} - 4 \cdot \frac{13 \cdot 12}{2} =
1326 - 312 =
1014
\end{equation}
The number of possible pairs is given by equation~\ref{eq:number_of_pairs}. The number of ways to draw a suit from a deck, given that the cards drawn will be from different suits, is given by:
\begin{equation}
\frac{78}{1014} \approx 0.07692
\end{equation}
The solution can also be found by reasoning that the probability of the second card drawn will be the same rank as the first card drawn, given that the second card drawn will be from a different suit, is given by:
\begin{equation}
\frac{4 - 1}{52 - 13} = \frac{3}{39} \approx 0.07692
\end{equation}
\end{enumerate}

\subsection{Conditional probability}

\textit{If the occurrence of B makes A more likely, does the occurrence of A make B more likely? Why?}\vspace{0.1cm}
We begin with the knowledge that the occurrence of B makes A more likely:

\begin{equation}
\label{eq:a_given_b_more_likely_than_a}
P(A \vert B) > P(A)
\end{equation}

The \textbf{product rule} states that:

\begin{equation}
P(A \land B) = P(A \vert B) \cdot P(B)
\qquad\text{and}\qquad
P(A \land B) = P(B \vert A) \cdot P(A)
\end{equation}

Equating the two right-hand sides and dividing by $P(B)$ gives us \textbf{Bayes' rule}:

\begin{equation}
\label{eq:bayes}
P(A \vert B) = \frac{P(B \vert A) \cdot P(A)}{P(B)}
\end{equation}

Replacing the left-hand side of the known inequality from equation~\ref{eq:a_given_b_more_likely_than_a} with the right-hand side of equation~\ref{eq:bayes} gives us the inequality:

\begin{equation}
\frac{P(B \vert A) \cdot P(A)}{P(B)} > P(A)
\end{equation}

Multiplying both sides by $P(B)$:

\begin{equation}
P(B \vert A) \cdot P(A) > P(A) \cdot P(B)
\end{equation}

Dividing both sides by $P(A)$:

\begin{equation}
\label{eq:b_given_a_more_likely_than_b}
P(B \vert A) > P(B)
\end{equation}

This derivation shows that if it is known that the probability of A given B is greater than the prior probability of A (equation~\ref{eq:a_given_b_more_likely_than_a}), it is also true that the probability of B given A is greater than the prior probability of B (equation~\ref{eq:b_given_a_more_likely_than_b}).

Intuitively this makes sense, since knowledge that a cause is more probable given the presence of a symptom would also make the probability of the symptom more likely given the presence of the cause.

\section{Bayesian Network Construction}

We first begin by listing clear causal links:

\begin{itemize}
\item \textit{Working parents}
\begin{itemize}
\item \textit{Household income}
\end{itemize}
\item \textit{Religion}
\begin{itemize}
\item \textit{Number of children} -- 
\item \textit{Working parents} -- The study \textit{``Religion, attitudes towards working mothers and wives' full-time employment''} (Guido Heineck 2004) performed with data from Austria, Germany, Italy, the UK, and the USA, found that the effect of religious affiliation leading to a lower likelihood of full-time working wives is weak, and that the attitudes of males' towards female employment has a much greater effect.\\Looking at numbers for Islam specifically, it can be seen that for countries such as Iran (98\% Muslim) or Iraq (97\% Muslim), the female labor force participation rate is 16\% and 14\% -- while Bangladesh (89\% Muslim) has a participation rate of 56\%.\\https://en.wikipedia.org/wiki/Female\_labor\_force\_in\_the\_Muslim\_world\\Given that religious affiliation lends to more traditional views, and that more traditional views lends to a more traditional family, there is likely to be some link between religion and number of working parents. The strength of this link will vary with the country of focus, but it seems clear that the number of working parents is not independent of religion.
\item \textit{Household income} -- The study \textit{``Religion and wealth: The role of religious affiliation and participation in early adult asset accumulation''} (Lisa A. Keister 2011) found clear links between religious affiliation and wealth, with adherents of Judaism and Episcopalianism attaining the most wealth, believers of Catholicism and mainline Protestants being in the middle, and conservative Protestants achieving the least wealth. The study also found that those who attend religious service achieved more wealth than those who do not. The same researcher states in another study \textit{``Conservative Protestants and Wealth: How Religion Perpetuates Asset Poverty''} (Lisa A. Keister 2008) that ``religion affects wealth indirectly through educational attainment, fertility, and female labour force participation'', but that there can also be direct causal links between religion and income. Direct effects can be explained as adherents to certain religious groups considering that ``money belongs to God'', being more prone to give away their money, and that they may seek to avoid accumulating wealth itself.\\This suggests that there are both indirect and direct causal links between religion and household income.
\item \textit{Fish-eating habits} -- Vegetarian diets are common in Dharmic based faiths such as Hinduism, Jainism, Sikhism, and Taoism, due to the principle of \textit{Ahimsa}, i.e. ``do no harm''.\\
Since there are large religious groups where vegetarian diets are common due to the religion, there is a causal link between religion and fish-eating habits; however for Christianity, Judaism, Islam and Atheism, there is likely no significant causal links.
    \item \textit{Drinking habits} -- Alcohol is not specifically forbidden by the Qur'an, however there is a consensus within Islamic theologians that alcohol is prohibited by Islam as it weakens the conscience of the believer. Some Christian groups such as Baptists and Methodists believe in abstinence from alcohol. Initiated Sikhs cannot drink alcohol, as it is an intoxicant. For Buddhism, alcohol violates the 5th of the \textit{Five Precepts}; ``I undertake the training rule to abstain from fermented drink that causes headlessness''.\\https://en.wikipedia.org/wiki/Religion\_and\_alcohol\\As several large religions have rules prohibiting or discouraging the use of alcohol, there is a direct causal link between religion and drinking habits.
\end{itemize}
\end{itemize}

\end{document}

