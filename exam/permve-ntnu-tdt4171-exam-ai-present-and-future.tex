\cardfrontfoot{AI: Present and Future}

\begin{flashcard}[Question]{What are \textbf{anytime algorithms}?}
\begin{center}
\textbf{Anytime algorithms} are algorithms whose output quality improves gradually over time, such that it has a reasonable decision ready whenever interrupted.
\end{center}
\end{flashcard}

\begin{flashcard}[Question]{What is \textbf{decision-theoretic meta-reasoning}?}
\begin{center}
\textbf{Decision-theoretic meta-reasoning} applies the theory of information value to the selection of computations, where the value of a computation depends on both its cost and its benefits.
\end{center}
\end{flashcard}

\begin{flashcard}[Question]{What are four possible agent specifications?}
\begin{center}
\begin{itemize}
\item \textbf{Perfect rationality:} acting to maximize expected utility given all available information from the environment. This has been judged as beeing too time-consuming and not a realistic goal.
\item \textbf{Calculative rationality:} eventually returning what would have been the rational choice when beginning deliberation. This is an interesting goal, but in many environments the right answer at the wrong time is not good enough.
\item \textbf{Bounded rationality:} deliberating only long enough to come up with a ``good enough'' answer, also known as ``satificing'', an approach advocated by Herbert Simon.
\item \textbf{Bounded optimality:} behaving as well as possible, given the computational resources available. The expected utility of the agent program for a \textbf{bounded optimal agent} is at least as high as the expected utility of any other agent program running on the same machine.
\end{itemize}
\end{center}
\end{flashcard}

\begin{flashcard}[Question]{How is \textbf{asymptotic bounded optimality} defined?}
\begin{center}
Suppose a program \textbf{P} is bounded optimal for a machine $M$ in a class of environments $E$, where the complexity of environments in $E$ is unbounded.

\medskip

The program \textbf{P'} is then \textbf{asymptotic bounded optimal} for $M$ in $E$ if it can outperform \textbf{P} by running on a machine $kM$ that is $k$ times faster (or larger) than $M$.
\end{center}
\end{flashcard}
