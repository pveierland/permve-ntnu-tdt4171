\cardfrontfoot{Philosophical Foundations}

\begin{flashcard}[Question]{What is the difference between \textbf{weak AI} and \textbf{strong AI}?}
\begin{center}
The \textbf{weak AI} hypothesis is the assertion that machines can \textit{act} intelligently.

\medskip

The \textbf{strong AI} hypothesis is the assertion that machines can \textit{actually} think.
\end{center}
\end{flashcard}

\begin{flashcard}[Question]{What are three central arguments against intelligent machines?}
\begin{itemize}
\item The argument from \textbf{disability}: a machine will never be able to ``be kind'', ``enjoy strawberries'', ``be creative''.
\item The \textbf{mathematical objection}: since Gödel proved that it is possible to construct true, but unprovable sentences in any sufficiently powerful formal systems, philosophers have argued that this makes machines mentally inferior to humans.
\item The argument from \textbf{informality} states that human behavior is to complex to be captured by rules, and because computers are governed by rules, it is therefore not possible for computers to achieve the complexity of human behavior. The problem of capturing everything in a set of logical rules is called the \textbf{qualification~problem}.
\end{itemize}
\end{flashcard}

\begin{flashcard}[Question]{}
\end{flashcard}

\begin{flashcard}[Question]{}
\end{flashcard}

\begin{flashcard}[Question]{}
\end{flashcard}

\begin{flashcard}[Question]{}
\end{flashcard}

\begin{flashcard}[Question]{}
\end{flashcard}

\begin{flashcard}[Question]{}
\end{flashcard}

\begin{flashcard}[Question]{}
\end{flashcard}

\begin{flashcard}[Question]{}
\end{flashcard}

\begin{flashcard}[Question]{}
\end{flashcard}
