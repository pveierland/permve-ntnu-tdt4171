\cardfrontfoot{Philosophical Foundations}

\begin{flashcard}[Question]{What is the difference between \textbf{weak AI} and \textbf{strong AI}?}
\begin{center}
The \textbf{weak AI} hypothesis is the assertion that machines can \textit{act} intelligently.

\medskip

The \textbf{strong AI} hypothesis is the assertion that machines can \textit{actually} think.
\end{center}
\end{flashcard}

\begin{flashcard}[Question]{What are three central arguments against machines \textit{acting} intelligently?}
\begin{itemize}
\item The argument from \textbf{disability}: a machine will never be able to ``be kind'', ``enjoy strawberries'', ``be creative''.
\item The \textbf{mathematical objection}: since Gödel proved that it is possible to construct true, but unprovable sentences in any sufficiently powerful formal systems, philosophers have argued that this makes machines mentally inferior to humans.
\item The argument from \textbf{informality} states that human behavior is to complex to be captured by rules, and because computers are governed by rules, it is therefore not possible for computers to achieve the complexity of human behavior. The problem of capturing everything in a set of logical rules is called the \textbf{qualification~problem}.
\end{itemize}
\end{flashcard}

\begin{flashcard}[Question]{What is \textbf{functionalism} and \textbf{biological naturalism}?}
\begin{center}
\textbf{Functionalism} is a theory stating that a mental state is any intermediate causal condition between input and output. Under this theory, any two systems with isomorphic causal processes would have the same mental state.

\medskip

\textbf{Biological naturalism} is a theory stating that mental states are high-level emergent features which are caused by low-level neurological processes in the neurons, and that it is the properties of the neurons which matter.
\end{center}
\end{flashcard}

\begin{flashcard}[Question]{What is \textbf{dualism}, \textbf{monism}, and the \textbf{mind-body problem}?}
\begin{center}
\textbf{Dualism} is a theory stating that the soul and body are distinct things, and that the soul is non-material.

\medskip

\textbf{Monism}, often called \textbf{materialism}, is a theory stating that everything consists of material objects, as opposed to \textbf{dualism}.

\medskip

The \textbf{mind-body problem} asks how mental states and processes are related to bodily states and processes, i.e. whether ``brain~states'' are ``mental~states''.
\end{center}
\end{flashcard}

\begin{flashcard}[Question]{What are \textbf{propositional attitudes}?}
\begin{center}
\textbf{Propositional attitudes}, also known as \textbf{intentional states}, are mental states which refer to some aspect of the external world, e.g. ``believing'', ``knowing'', or ``desiring''.
\end{center}
\end{flashcard}

\begin{flashcard}[Question]{What are the \textbf{wide content} and the \textbf{narrow content} views?}
\begin{center}
The \textbf{wide content} view holds that beliefs held must be judged from an outside observer, such that a ``brain-in-a-vat'' experiencing desire for a hamburger is diffent from a ``normal person'' desiring a hamburger.

\medskip

The \textbf{narrow content} view holds that beliefs are judged from an internal and subjective point of view. Under this view the two experiences would be the same.

\end{center}
\end{flashcard}

\begin{flashcard}[Question]{What is \textbf{qualia}?}
\begin{center}
\textbf{Qualia}, or \textbf{intrinsic experiences}, refers to the subjective experience of a certain sensory input.

\medskip

For example one person seeing the color ``red'' may have the same experience which another person has when they see the color ``green''. They both agree that they are seeing the color ``red'', but their subjective experiences feel different.
\end{center}
\end{flashcard}
